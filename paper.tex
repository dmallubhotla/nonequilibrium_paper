\documentclass{article}

%other packages
\usepackage{amsmath}
\usepackage{amssymb}
\usepackage{physics}

\usepackage[
	style=phys, articletitle=false, biblabel=brackets, chaptertitle=false, pageranges=false, url=true
]{biblatex}

\usepackage{graphicx}
\usepackage{todonotes}
\usepackage{siunitx}

\usepackage{cleveref}

\title{Evanescent-wave Johnson noise from a non-equilibrium superconductor}

\addbibresource{./bibliography.bib}

\graphicspath{{./figures/}}

\newcommand{\pf}{p_{\mathrm{F}}}
\newcommand{\vf}{v_{\mathrm{F}}}
\newcommand{\epsF}{\epsilon_{\mathrm{F}}}
\newcommand{\debye}{\omega_{\mathrm{D}}}
\newcommand{\corr}{\mu^{\ast}}
\newcommand{\dof}{\rho}

\begin{document}

\maketitle

\section{Methods \label{sec:methods}}

\subsection{Gap calculation \label{subsec:gapcalc}}

We must calculate the superconducting gap in the case where a material is kept out of equilibrium by quasiparticle injection.
This quasiparticle injection occurs via multiple physical mechanisms, such as shining photons on a superconducting surface as in Gao et. al\cite{Gao2008}, or applying a bias voltage as studied in Catelani et. al\cite{Catelani2010}.
Both Gao and Catelani obtain via different methods results analogous to Owen and Scalapino\cite{OwenScalapino}, which we summarise here.\todo{Revisit with more detail about how Catelani is probably a more rigorous derivation of the OS results.}

Owen and Scalapino derive the usual BCS gap equation, but by constraining quasiparticle number end up with an effective quasiparticle distribution function
\begin{equation}
	f_k = \frac{1}{1 + \exp[\beta \left(E_k - \corr \right)]}.
\end{equation}
This results in a modified gap equation:\todo{add more derivation so that people don't have to go back to BCS to understand.}
\begin{equation}
	\left[ N(0) V \right]^{-1} = \int_{- \omega_D}^{\omega_D} \frac{\dd{\epsilon_k}}{\sqrt{\Delta^2 + \epsilon_k^2}} \tanh{\frac{\sqrt{\Delta^2 + \epsilon_k^2} - \corr}{2 T}}. \label{eq:gap}
\end{equation}
Owen and Scalapino also characterise the excess quasiparticle density $n$ in units of $4 N(0) \Delta_0$:
\begin{equation}
	n = \frac{1}{\Delta_0} \int_0^\infty \dd{\epsilon} \left( \frac{1}{1 + \exp(\frac{\sqrt{\Delta^2 + \epsilon_k^2} - \corr}{T})} - \frac{1}{1 + \exp(\frac{\sqrt{\Delta^2 + \epsilon_k^2}}{T})} \right). \label{eq:n}
\end{equation}

We are going to be interested in the case where we have a fixed $n$. \todo{why? Assumption for the case with a bias voltage that we are measuring some the current which will be $\propto n$, but for photons shining it's harder to justify?}
This means that \eqref{eq:gap} and \eqref{eq:n} form two coupled integral equations in $\Delta$ and $\corr$, to be solved numerically.\todo{Include gap calculation plots here.}

These results are constrained further by the material's energetics;
as Owen and Scalapino discuss, the solutions of \eqref{eq:gap} and \eqref{eq:n} may not be energetically favourable, and for increasing $n$ the critical temperature $T_c(n)$ will decrease.\cite{OwenScalapino}
Determining this decrease requires a model of the particular method by which the quasiparticle injection occurs, as the normal state free energy may change as well.
However, here we use Owen and Scalapino's approximate relation $T_c(n) \approx (1 - 4n) T_c(n = 0)$, as that is sufficiently descriptive for this study.

\subsection{Electromagnetic response}

To find the noise outside the surface, we must find the electromagnetic response function of the superconductor using the gap found above.
Because we are interested in a wide range of impurity concentrations in the supercondcutor, we use the electromagnetic response function first derived by Nam\cite{Nam1967}, which extends the previous models by Mattis and Bardeen~\cite{Mattis} and Abrikosov, Dzyaloshinskii and Gorkov\cite{AGD}.

The response function in Nam depends on the material's plasma frequency $\omega_p$, an impurity collision frequency $\tau$, the Fermi velocity $\vf$, and the gap $\Delta$.
For our noise calculation, we will need this function over the Fourier momentum $q$ and frequency $\omega$.
To simplify the notation, we define the following dimensionless quantities, keeping the variable $\mu$:
\begin{align}
	\xi &= \frac{\omega}{\Delta} \\
	\xi' &= \frac{\omega'}{\Delta} \\
	\nu &= \frac{1}{\tau \Delta} \\
	\kappa &= \frac{q \vf}{\Delta} \\
	t &= \frac{T}{\Delta} \\
	\sigma_0 &= \frac{n e^2}{m \Delta} \\
	\mu &= \frac{\corr}{\Delta}
\end{align}
Because our model includes modified quasiparticle distribution functions, we must modify these terms in the response function as well as the gap.
These appear in the $\tanh$ functions as an effective chemical potential.\todo{Write this out a bit more, sketching out the actual derivation.}
Then the expression for the conductivity at arbitrary momentum and frequency in Nam\cite{Nam1967}, with the modified distribution function, becomes:
\begin{equation}
	\sigma(\kappa, \xi) = -i \frac{3 \sigma_0}{4} \frac{1}{\xi}\left[\int_{1 - \xi}^{1}\dd{\xi} \tanh(\frac{\xi + \xi' - \mu}{2 t}) I_1 + \int_{1}^{\infty} \dd{\xi'} \left( \tanh(\frac{\xi + \xi' - \mu}{2t}) I_1  - \tanh(\frac{\xi' - \mu}{2t})I_2 \right) \right] \label{eq:nam}
\end{equation}
with
\begin{align}
	I_1 &= F(\kappa, \Re[\sqrt{(\xi + \xi')^2 - 1} - \sqrt{\xi'^2 - 1}]) (g + 1) \nonumber\\
	&\quad + F(\kappa, \Re[-\sqrt{(\xi + \xi')^2 - 1} - \sqrt{\xi'^2 - 1}]) (g - 1) \\
	I_2 &= F(\kappa, \Re[\sqrt{(\xi + \xi')^2 - 1} - \sqrt{\xi'^2 - 1}]) (g + 1) \nonumber\\
	&\quad + F(\kappa, \Re[\sqrt{(\xi + \xi')^2 - 1} + \sqrt{\xi'^2 - 1}]) (g - 1) \\
	F(\kappa, E) &= \frac{1}{\kappa} \left[2 S(E) + (1 - S(E)^2)\ln(\frac{S(E) + 1}{S(E) - 1})\right]  \\
	S(\kappa, E) &= \frac{1}{\kappa} \left(E - i \left(\Im[\sqrt{(\xi + \xi')^2 - 1} + \sqrt{\xi'^2 - 1}] + 2 \nu \right) \right) \\
	g  &= \frac{\xi' \left( \xi + \xi'\right) + 1}{\sqrt{\xi'^2 - 1}\sqrt{(\xi + \xi')^2 - 1}}
\end{align}
We note here the similarity between \eqref{eq:nam} and the response functions in Catelini et. al\cite{Catelani2010}.\todo{develop more. It will probably be very interesting to actually show how their expressions match ours in the first approximation where our $q = 0$ and their quasiparticle distribution function is approximated.}

\subsection{Noise calculation}

\section{Results}

\printbibliography
\listoftodos

\end{document}
